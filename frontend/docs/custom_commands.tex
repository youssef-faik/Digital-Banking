% Custom commands for consistent formatting
\newcommand{\sectiondivider}{
    \begin{center}
        \textcolor{primary}{\rule{0.3\textwidth}{1pt}}
        \quad\faChevronDown\quad
        \textcolor{primary}{\rule{0.3\textwidth}{1pt}}
    \end{center}
}

\newcommand{\featureheading}[1]{
    \vspace{0.5em}
    \noindent\textcolor{primary}{\textbf{\sffamily\large #1}}
    \vspace{0.5em}
}

\newcommand{\technote}[1]{
    \begin{secondarybox}[title=Note technique]
        \sffamily\small #1
    \end{secondarybox}
}

\newcommand{\keypoint}[1]{
    \begin{primarybox}[title=Point clé]
        \sffamily #1
    \end{primarybox}
}

% Command for code snippets
\newcommand{\codesnippet}[2]{
    \begin{tcolorbox}[
        colback=darktext!5!white,
        colframe=darktext!50!white,
        arc=3pt,
        title=\textcolor{darktext}{\textbf{\sffamily #1}},
        fonttitle=\small\sffamily,
        boxrule=0.5pt,
        left=5pt,right=5pt,top=5pt,bottom=5pt
    ]
    \small\ttfamily #2
    \end{tcolorbox}
}

% Command for API endpoint documentation
\newcommand{\apiendpoint}[4]{
    \begin{tcolorbox}[
        enhanced,
        colback=lightgray,
        colframe=tertiary,
        arc=3pt,
        title=\textcolor{white}{\textbf{\sffamily #1 #2}},
        fonttitle=\sffamily,
        coltitle=white,
        colbacktitle=tertiary,
        boxrule=0.5pt,
        left=5pt,right=5pt,top=5pt,bottom=5pt
    ]
    \small\sffamily
    \textbf{Description:} #3 \\
    \textbf{Réponse:} \texttt{#4}
    \end{tcolorbox}
}

% Command for feature showcase
\newcommand{\featureshowcase}[3]{
    \begin{minipage}{\textwidth}
        \begin{minipage}{0.65\textwidth}
            \begin{secondarybox}[title=#1]
                \sffamily #2
            \end{secondarybox}
        \end{minipage}
        \hfill
        \begin{minipage}{0.3\textwidth}
            \begin{imagebox}
                \includegraphics[width=\textwidth]{#3}
            \end{imagebox}
        \end{minipage}
    \end{minipage}
}

% Command for technology badge
\newcommand{\techbadge}[2]{
    \tikz[baseline=(char.base)]{
        \node[
            fill=#2!10,
            draw=#2,
            rounded corners=3pt,
            inner sep=2pt,
            text=#2
        ] (char) {#1};
    }
}

% Command for step-by-step process
\newcommand{\processstep}[2]{
    \begin{tcolorbox}[
        enhanced,
        colback=white,
        colframe=secondary,
        leftrule=3pt,
        rightrule=0pt,
        toprule=0pt,
        bottomrule=0pt,
        arc=0pt,
        left=8pt
    ]
        \textbf{\sffamily\textcolor{secondary}{Étape #1:}} \sffamily #2
    \end{tcolorbox}
}

% Command for comparison table header row
\newcommand{\comparisonheader}[3]{
    \rowcolor{primary!20}
    \textbf{\sffamily #1} & \textbf{\sffamily #2} & \textbf{\sffamily #3} \\
}

% Command for bullet with icon
\newcommand{\iconbullet}[2]{
    \noindent\textcolor{primary}{#1} \sffamily #2 \\
}

% Command for feature card
\newcommand{\featurecard}[4]{
    \begin{tcolorbox}[
        enhanced,
        colback=#3!5,
        colframe=#3,
        arc=5pt,
        title=\textcolor{white}{#1 \textbf{\sffamily #2}},
        fonttitle=\sffamily,
        coltitle=white,
        colbacktitle=#3,
        boxrule=0.5pt,
        left=8pt,right=8pt,top=8pt,bottom=8pt
    ]
        \sffamily\small #4
    \end{tcolorbox}
}
}

% Command for timeline entry
\newcommand{\timelineentry}[3]{
    \begin{minipage}{\textwidth}
        \begin{tikzpicture}
            \filldraw[fill=#3!20, draw=#3] (0,0) circle (0.4cm);
            \node at (0,0) {\textcolor{#3}{\Large\bfseries #1}};
            \draw[thick, #3] (0.4,0) -- (0.7,0);
            \node[anchor=west, text width=\textwidth-1cm] at (0.8,0) {\textbf{\sffamily\textcolor{#3}{#2}}};
        \end{tikzpicture}
    \end{minipage}
    \vspace{0.2cm}
}

% Command for technology stack item
\newcommand{\techstack}[3]{
    \begin{tikzpicture}
        \node[
            draw=#2,
            fill=#2!5,
            rounded corners=5pt,
            minimum width=3cm,
            minimum height=1.5cm
        ] {
            \begin{tabular}{c}
                \textcolor{#2}{#1} \\
                \textbf{\sffamily\small #3}
            \end{tabular}
        };
    \end{tikzpicture}
}
